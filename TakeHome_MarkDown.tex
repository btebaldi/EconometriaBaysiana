% Options for packages loaded elsewhere
\PassOptionsToPackage{unicode}{hyperref}
\PassOptionsToPackage{hyphens}{url}
%
\documentclass[
]{article}
\usepackage{lmodern}
\usepackage{amssymb,amsmath}
\usepackage{ifxetex,ifluatex}
\ifnum 0\ifxetex 1\fi\ifluatex 1\fi=0 % if pdftex
  \usepackage[T1]{fontenc}
  \usepackage[utf8]{inputenc}
  \usepackage{textcomp} % provide euro and other symbols
\else % if luatex or xetex
  \usepackage{unicode-math}
  \defaultfontfeatures{Scale=MatchLowercase}
  \defaultfontfeatures[\rmfamily]{Ligatures=TeX,Scale=1}
\fi
% Use upquote if available, for straight quotes in verbatim environments
\IfFileExists{upquote.sty}{\usepackage{upquote}}{}
\IfFileExists{microtype.sty}{% use microtype if available
  \usepackage[]{microtype}
  \UseMicrotypeSet[protrusion]{basicmath} % disable protrusion for tt fonts
}{}
\makeatletter
\@ifundefined{KOMAClassName}{% if non-KOMA class
  \IfFileExists{parskip.sty}{%
    \usepackage{parskip}
  }{% else
    \setlength{\parindent}{0pt}
    \setlength{\parskip}{6pt plus 2pt minus 1pt}}
}{% if KOMA class
  \KOMAoptions{parskip=half}}
\makeatother
\usepackage{xcolor}
\IfFileExists{xurl.sty}{\usepackage{xurl}}{} % add URL line breaks if available
\IfFileExists{bookmark.sty}{\usepackage{bookmark}}{\usepackage{hyperref}}
\hypersetup{
  pdftitle={Econometria Baysiana - Take Home Exam},
  hidelinks,
  pdfcreator={LaTeX via pandoc}}
\urlstyle{same} % disable monospaced font for URLs
\usepackage[margin=1in]{geometry}
\usepackage{color}
\usepackage{fancyvrb}
\newcommand{\VerbBar}{|}
\newcommand{\VERB}{\Verb[commandchars=\\\{\}]}
\DefineVerbatimEnvironment{Highlighting}{Verbatim}{commandchars=\\\{\}}
% Add ',fontsize=\small' for more characters per line
\usepackage{framed}
\definecolor{shadecolor}{RGB}{248,248,248}
\newenvironment{Shaded}{\begin{snugshade}}{\end{snugshade}}
\newcommand{\AlertTok}[1]{\textcolor[rgb]{0.94,0.16,0.16}{#1}}
\newcommand{\AnnotationTok}[1]{\textcolor[rgb]{0.56,0.35,0.01}{\textbf{\textit{#1}}}}
\newcommand{\AttributeTok}[1]{\textcolor[rgb]{0.77,0.63,0.00}{#1}}
\newcommand{\BaseNTok}[1]{\textcolor[rgb]{0.00,0.00,0.81}{#1}}
\newcommand{\BuiltInTok}[1]{#1}
\newcommand{\CharTok}[1]{\textcolor[rgb]{0.31,0.60,0.02}{#1}}
\newcommand{\CommentTok}[1]{\textcolor[rgb]{0.56,0.35,0.01}{\textit{#1}}}
\newcommand{\CommentVarTok}[1]{\textcolor[rgb]{0.56,0.35,0.01}{\textbf{\textit{#1}}}}
\newcommand{\ConstantTok}[1]{\textcolor[rgb]{0.00,0.00,0.00}{#1}}
\newcommand{\ControlFlowTok}[1]{\textcolor[rgb]{0.13,0.29,0.53}{\textbf{#1}}}
\newcommand{\DataTypeTok}[1]{\textcolor[rgb]{0.13,0.29,0.53}{#1}}
\newcommand{\DecValTok}[1]{\textcolor[rgb]{0.00,0.00,0.81}{#1}}
\newcommand{\DocumentationTok}[1]{\textcolor[rgb]{0.56,0.35,0.01}{\textbf{\textit{#1}}}}
\newcommand{\ErrorTok}[1]{\textcolor[rgb]{0.64,0.00,0.00}{\textbf{#1}}}
\newcommand{\ExtensionTok}[1]{#1}
\newcommand{\FloatTok}[1]{\textcolor[rgb]{0.00,0.00,0.81}{#1}}
\newcommand{\FunctionTok}[1]{\textcolor[rgb]{0.00,0.00,0.00}{#1}}
\newcommand{\ImportTok}[1]{#1}
\newcommand{\InformationTok}[1]{\textcolor[rgb]{0.56,0.35,0.01}{\textbf{\textit{#1}}}}
\newcommand{\KeywordTok}[1]{\textcolor[rgb]{0.13,0.29,0.53}{\textbf{#1}}}
\newcommand{\NormalTok}[1]{#1}
\newcommand{\OperatorTok}[1]{\textcolor[rgb]{0.81,0.36,0.00}{\textbf{#1}}}
\newcommand{\OtherTok}[1]{\textcolor[rgb]{0.56,0.35,0.01}{#1}}
\newcommand{\PreprocessorTok}[1]{\textcolor[rgb]{0.56,0.35,0.01}{\textit{#1}}}
\newcommand{\RegionMarkerTok}[1]{#1}
\newcommand{\SpecialCharTok}[1]{\textcolor[rgb]{0.00,0.00,0.00}{#1}}
\newcommand{\SpecialStringTok}[1]{\textcolor[rgb]{0.31,0.60,0.02}{#1}}
\newcommand{\StringTok}[1]{\textcolor[rgb]{0.31,0.60,0.02}{#1}}
\newcommand{\VariableTok}[1]{\textcolor[rgb]{0.00,0.00,0.00}{#1}}
\newcommand{\VerbatimStringTok}[1]{\textcolor[rgb]{0.31,0.60,0.02}{#1}}
\newcommand{\WarningTok}[1]{\textcolor[rgb]{0.56,0.35,0.01}{\textbf{\textit{#1}}}}
\usepackage{graphicx,grffile}
\makeatletter
\def\maxwidth{\ifdim\Gin@nat@width>\linewidth\linewidth\else\Gin@nat@width\fi}
\def\maxheight{\ifdim\Gin@nat@height>\textheight\textheight\else\Gin@nat@height\fi}
\makeatother
% Scale images if necessary, so that they will not overflow the page
% margins by default, and it is still possible to overwrite the defaults
% using explicit options in \includegraphics[width, height, ...]{}
\setkeys{Gin}{width=\maxwidth,height=\maxheight,keepaspectratio}
% Set default figure placement to htbp
\makeatletter
\def\fps@figure{htbp}
\makeatother
\setlength{\emergencystretch}{3em} % prevent overfull lines
\providecommand{\tightlist}{%
  \setlength{\itemsep}{0pt}\setlength{\parskip}{0pt}}
\setcounter{secnumdepth}{-\maxdimen} % remove section numbering

\title{Econometria Baysiana - Take Home Exam}
\author{}
\date{\vspace{-2.5em}13/02/2020}

\begin{document}
\maketitle

\hypertarget{question}{%
\subsection{Question}\label{question}}

Consider the regression model \(y_i = x_i'\beta + \epsilon_i0\)
\(\epsilon_i | \lambda_i ,\sigma^2 \sim N(0, \lambda_i\sigma^2)\) where
\(x_i = (1, x_{i1}, \dots, x_{iq})'\) is a p-dimensional vector of
regressors (constant plus q attributes or characteristics) and the
following hierarchical prior for the scale-mixing variables
\(\lambda_i\): \begin{equation}
\lambda_1, \dots, \lambda_n \sim \text{iid Exponential(1/2)}
\end{equation}

\hypertarget{part-a}{%
\subsubsection{PART A)}\label{part-a}}

We will show that \begin{equation}
p\left(\epsilon_{i} | \sigma^{2}\right)=\int_{0}^{\infty} p\left(\epsilon_{i} | \lambda_{i}, \sigma^{2}\right) p\left(\lambda_{i}\right) d \lambda_{i}=\frac{1}{2 \sigma} \exp \left\{-\frac{\left|\epsilon_{i}\right|}{\sigma}\right\}
\end{equation}

First consider \(p\left(\epsilon_{i} | \sigma^{2}\right)\), then we must
have that: \begin{align}
p\left(\epsilon_{i} | \sigma^{2}\right) &=\int_{0}^{\infty} p\left(\epsilon_{i} | \lambda_{i}, \sigma^{2}\right) p\left(\lambda_{i}\right) d \lambda_{i} \\
&=\int_{0}^{\infty}\left(2 \pi \lambda_{i} \sigma^{2}\right)^{-1 / 2} \exp \left[-\epsilon_{i}^{2} /\left(2 \lambda_{i} \sigma^{2}\right)\right](1 / 2) \exp \left(-\lambda_{i} / 2\right) d \lambda_{i} \\
&=(1 / 2)\left(2 \pi \sigma^{2}\right)^{-1 / 2} \int_{0}^{\infty} \lambda_{i}^{-1 / 2} \exp \left[-(1 / 2)\left(\lambda_{i}+\left[\epsilon_{i} / \sigma\right]^{2} \lambda_{i}^{-1}\right)\right] d \lambda_{i}
\end{align}

Now, make a change of variable and let \(\psi_i = \lambda_i^{1/2}\). We
can then express this integral as \[\begin{equation}
p\left(\epsilon_{i} | \sigma^{2}\right)=\left(2 \pi \sigma^{2}\right)^{-1 / 2} \int_{0}^{\infty} \exp \left(-[1 / 2]\left(\psi_{i}^{2}+\left[\epsilon_{i} / \sigma\right]^{2} \psi_{i}^{-2}\right)\right) d \psi_{i}
\end{equation}\]

The integral in above can be evaluated analytically. Using the following
result from Andrews and Mallows (1974) \[
\int_{0}^{\infty} \exp \left\{-0.5\left(a^{2} u^{2}+b^{2} u^{-2}\right)\right\} d u=\left(\frac{\pi}{2 a^{2}}\right)^{1/2} \exp \{-|a b|\}
\] Then \(a=1\), \(b=\epsilon_i/\sigma\), and \(u = \psi_i\)

\[
p\left(\epsilon_{i} | \sigma^{2}\right)=\left(2 \pi \sigma^{2}\right)^{-1 / 2} \left(\frac{\pi}{2}\right)^{1/2} \exp\{-|\epsilon_i/\sigma|\} =\frac{1}{2\sigma} \exp\left\{-\frac{|\epsilon_i|}{\sigma}\right\}
\]

\hypertarget{part-b}{%
\subsubsection{PART B)}\label{part-b}}

Let \(y = (y_1, \dots, y_n)'\) and \(X = (x_1, \dots, x_n)'\). Where we
have the the following independent priors for
\(\beta \sim N(\beta_0, V_0)\) and
\(\sigma^2 \sim IG\left(\frac{\nu_0}{2}, \frac{\nu_0\sigma_0^2}{2}\right)\).
Also let \(\mathcal{D} = \{y;X\}\), and \(\Lambda\) be denoted as
follows \[
\Lambda  = \begin{bmatrix}\lambda_1 & 0 & \dots & 0 \\ 0 & \lambda_2 & \dots & 0 \\ 0 & \vdots & \ddots & 0 \\ 0 & 0 & 0 & \lambda_n \end{bmatrix}
\] To implement the Gibbs sampler we need to obtain the complete
posterior conditionals for the parameters \(\beta\), \(\sigma^2\), and
\(\{\lambda_i\}_{i=1}^{n}\) and cycle through the posteriors conditional
distributions. The joint posterior distribution is given as
\begin{equation}
p\left(\beta,\left\{\lambda_{i}\right\}, \sigma^{2} | y\right) \propto\left[\prod_{i=1}^{n} \phi\left(y_{i} ; x_{i} \beta, \lambda_{i} \sigma^{2}\right) p\left(\lambda_{i}\right)\right] p(\beta) p\left(\sigma^{2}\right)
\end{equation}

We know that traditional GLS have that
\(\beta = (X^T\Lambda^{-1}X)^{-1}X^T\Lambda^{-1}y\). If
\(\beta \sim N(\beta_0, V_0)\), and
\(\sigma^2 \sim IG\left(\frac{\nu_0}{2}, \frac{\nu_0 \sigma_0^2}{2}\right)\)
then from the joint posterior, the following complete conditional
posterior distributions are obtained:

\hypertarget{the-beta-conditional-distribution-beta-lambda_i-sigma2-mathcald}{%
\paragraph{\texorpdfstring{The \(\beta\) conditional distribution
(\(\beta | \{\lambda_i\}, \sigma^2, \mathcal{D}\))}{The \textbackslash beta conditional distribution (\textbackslash beta \textbar{} \textbackslash\{\textbackslash lambda\_i\textbackslash\}, \textbackslash sigma\^{}2, \textbackslash mathcal\{D\})}}\label{the-beta-conditional-distribution-beta-lambda_i-sigma2-mathcald}}

\[
\begin{align}
p\left(y | X, \beta, \sigma^{2}, \{\lambda_i\}\right) 
&\propto \exp \left\{-\frac{1}{2 \sigma^{2}} \sum_{i=1}^{n} \hat\epsilon_i^2 \right\} \\
&\propto \exp \left\{-\frac{1}{2 \sigma^{2}}\left[y^{T} y-2 \beta^{T} X^{T}\Lambda^{-1} y+\beta^{T} X^{T}\Lambda^{-1} X \beta\right]\right\} \\
p\left(\beta | \{\lambda_i\}, \sigma^2, \mathcal{D}\right) &\propto p\left(y | X, \beta, \sigma^{2}\right) \times p(\beta) \\
&\propto \exp \left\{-\frac{1}{2\sigma^2}\left(-2 \beta^{T} X^{T}\Lambda^{-1} y +\beta^{T} X^{T}\Lambda^{-1} X \beta \right)-\frac{1}{2}\left(-2 \beta^{T} V_{0}^{-1} \beta_{0}+\beta^{T} V_{0}^{-1} \beta\right)\right\} \\
&\propto 
\exp \left\{-\frac{1}{2}\left(-2 \beta^{T} (X^{T}\Lambda^{-1} y/\sigma^2 +V_{0}^{-1} \beta_{0})+ \beta^{T} (X^{T}\Lambda^{-1} X/\sigma^2 +V_{0}^{-1} ) \beta \right)
\right\}
\end{align}
\] we recognize this as being proportional to a multivariate normal
density, with \[
\begin{align}
\beta | \{\lambda_i\}, \sigma^2, \mathcal{D} \sim N(\beta_1, V_1) \\
\beta_{1}=V_{1}\left(X^{\prime} \Lambda^{-1} y/\sigma^2+V_{0}^{-1} \beta_{0}\right) \qquad V_{1}=\left(X^{\prime} \Lambda^{-1} X / \sigma^{2}+V_{0}^{-1}\right)^{-1}
\end{align}
\]

\hypertarget{the-sigma2-conditional-distribution-sigma2-lambda-mathcald-beta}{%
\paragraph{\texorpdfstring{The \(\sigma^2\) conditional distribution
(\(\sigma^2| \Lambda, \mathcal{D}, \beta\))}{The \textbackslash sigma\^{}2 conditional distribution (\textbackslash sigma\^{}2\textbar{} \textbackslash Lambda, \textbackslash mathcal\{D\}, \textbackslash beta)}}\label{the-sigma2-conditional-distribution-sigma2-lambda-mathcald-beta}}

As in most normal sampling problems, the semiconjugate prior
distribution for \(\sigma^2\) is an inverse-gamma distribution. Letting
\(\gamma = 1/\sigma^2\) be the measurement precision, this implies that
\(\gamma \sim G\left(\frac{\nu_0}{2}, \frac{\nu_0\sigma_0^2}{2}\right)\)
then \[
\begin{align}
p(\gamma | \mathcal{D}, \beta) & \propto p(\gamma) p(y | X, \beta, \gamma) \\
& \propto\left[\gamma^{\nu_{0} / 2-1} \exp \left\{-\gamma \times \frac{ \nu_{0} \sigma_{0}^{2}}{2}\right\}\right] \times\left[\gamma^{\frac{n}{2}} \exp \left\{- \frac{\gamma  }{2} \sum_{i=1}^{n} \hat\epsilon_i^2\right\}\right] \\
& \propto\gamma^{\left(\nu_{0}+n\right) / 2-1} \exp \left\{-\gamma\left[\nu_{0} \sigma_{0}^{2}+\sum_{i=1}^{n} \hat\epsilon_i^2\right] / 2\right\}
\end{align}
\] which we recognize as a gamma density, so that \[
\sigma^2| \Lambda, \mathcal{D}, \beta \sim IG\left(\frac{\nu_0 + n}{2}, \frac{\nu_0 \sigma_0^2 + \sum_{i=1}^{n} \hat\epsilon_i^2}{2}\right)
\] Recall that
\(\sum_{i=1}^{n} \hat\epsilon_i^2 = (y - X\beta)^T\Lambda^{-1}(y-x\beta)\)
\[
\begin{align}
\sigma^2| \Lambda, \mathcal{D}, \beta &\sim IG\left(\frac{\nu_0 + n}{2}, \frac{\nu_0 \sigma_0^2 + (y - X\beta)^T\Lambda^{-1}(y-x\beta)}{2}\right) \\
\sigma^2| \Lambda, \mathcal{D}, \beta &\sim IG\left(\frac{\nu_1}{2}, \frac{\nu_1 \sigma_1^2}{2}\right) \\
&\nu_1 = \nu_0 + n \qquad \nu_1\sigma_1^2 = \nu_0\sigma_0^2 + (y - X\beta)^T\Lambda^{-1}(y-x\beta)
\end{align}
\]

\hypertarget{the-lambda_i-conditional-distribution-lambda_i-beta-sigma2-mathcald}{%
\paragraph{\texorpdfstring{The \(\lambda_{i}\) conditional distribution
(\(\lambda_{i} | \beta, \sigma^{2}, \mathcal{D}\))}{The \textbackslash lambda\_\{i\} conditional distribution (\textbackslash lambda\_\{i\} \textbar{} \textbackslash beta, \textbackslash sigma\^{}\{2\}, \textbackslash mathcal\{D\})}}\label{the-lambda_i-conditional-distribution-lambda_i-beta-sigma2-mathcald}}

Lastly we have that \begin{align}
p\left(\lambda_{i} | \beta, \sigma^{2}, y_{i}, x_{i}\right) &\propto p\left( y_{i}| \lambda_{i}, \beta, \sigma^{2}, x_{i}\right) p(\lambda_{i}) \\
p\left(\lambda_{i} | \beta, \sigma^{2}, y_{i}, x_{i}\right) &\propto \frac{1}{\sqrt{2\pi \sigma^2 \lambda} }\exp \left\{-\frac{1}{2}\left(\left(y_{i}-x_{i}^{T} \beta\right)^{2} \sigma^{-2}\lambda_{i}^{-1}\right)\right\} \exp \left\{-0.5 \lambda_{i}\right\} \\
p\left(\lambda_{i} | \beta, \sigma^{2}, y_{i}, x_{i}\right) &\propto \lambda^{-1 / 2} \exp \left\{-0.5 \lambda_{i}\right\} 
\exp \left\{-0.5\left(\left(\frac{y_{i}-x_{i}^{\prime} \beta}{\sigma}\right)^{2} \lambda_{i}^{-1}\right)\right\} \\
p\left(\lambda_{i} | \beta, \sigma^{2}, y_{i}, x_{i}\right) &\propto \lambda^{-1 / 2} \exp \left\{-0.5\left(\lambda_{i}+\left(\frac{y_{i}-x_{i}^{\prime} \beta}{\sigma}\right)^{2} \lambda_{i}^{-1}\right)\right\}
\end{align}

We claim that this distribution is of the generalized inverse Gaussian
(GIG) form. Following Shuster (1968), Michael, et. al.~(1976), and
Carlin and Polson (1991), we outline a strategy for obtaining a draw
from this GIG density.

We say that \(x\) follows an inverse Gaussian distribution
(\(x \sim invGauss(\psi, \mu)\)) if \begin{equation}
p(x | \psi, \mu) \propto x^{-3 / 2} \exp \left(-\frac{\psi(x-\mu)^{2}}{2 x \mu^{2}}\right), \quad x>0
\end{equation}

Now, let \(z = x^{−1}\). It follows by a change of variables that
\begin{align}
p(z | \psi, \mu) &\propto z^{-2} z^{3 / 2} \exp \left(-\frac{\psi\left(z^{-1}-\mu\right)^{2}}{2 z^{-1} \mu^{2}}\right) \\
& \propto z^{-1 / 2} \exp \left(-\frac{\psi}{2}\left[z+\mu^{-2} z^{-1}\right]\right)
\end{align}

Then notice that the posterior conditional for \(\lambda_i\), follows
that the reciprocal of an \(invGauss(1, |\sigma/(y_i - x_i\beta)|)\).
\textbf{This means that a draw of \(\lambda_i\) can be done by inverting
a draw from the inverse Gaussian distribution.} Then, the only step is
to draw from the inverse Gaussian distribution.

Shuster (1968) notes that if \(x\) has the inverse Gaussian density,
then \(\psi(x − \mu)^2/x\mu^2 \sim \chi^2(1)\), a chi-square
distribution with one degree of freedom. Let
\(\nu_2 = \psi(x − \mu)^2/x\mu^2\), them the roots of \(nu_2\), denoted
here as \(x_1\) and \(x_2\) are obtained as \begin{align}
&x_{1}=\mu+\frac{\mu^{2} \nu_{2}}{2 \psi}-\frac{\mu}{2 \psi} \sqrt{4 \mu \psi \nu_{2}+\mu^{2} \nu_{2}^{2}}\\
&x_{2}=\mu^{2} / x_{1}
\end{align}

Michael et al.~(1976) use this idea to show that one can obtain a draw
from the inverse Gaussian \((\psi, \mu)\) density by first drawing
\(\nu_2 \sim \chi^2(1)\), calculating the roots \(x_1\) and \(x_2\) from
the preceding equations, and then setting \(x\) equal to \(x_1\) with
probability \(\mu/(\mu + x_1)\) and equal to x2 with probability
\(x_1/(\mu + x_1)\).

\hypertarget{part-c}{%
\subsubsection{PART C)}\label{part-c}}

We now simulate \(n = 200\) observations from the above linear
regression with double exponential errors model, where
\(\beta = (0, 1, 2, 3)'\), \(\sigma^2 = 1\) and \(x_{ij} \sim N(0, 1)\).
Afther the simulation we implement the above MCMC scheme and produce
posterior summaries of the main parameters. We also try to answer if the
simple MH algorithm to sample \(\lambda_i\) be reasonable for your
simulated data or the full-fledge Gibbs sampler performs better.

We first load libraries and clear any variables to start with a clear
enviroment.

\begin{Shaded}
\begin{Highlighting}[]
\CommentTok{# Libraries}
\KeywordTok{library}\NormalTok{(statmod) }\CommentTok{# Used for Inverse Gausian}
\KeywordTok{library}\NormalTok{(nimble)  }\CommentTok{# Used for double exponential}
\KeywordTok{library}\NormalTok{(tictoc)  }\CommentTok{# Used for Time Evaluation}

\CommentTok{# Clear Vars}
\KeywordTok{rm}\NormalTok{(}\DataTypeTok{list =} \KeywordTok{ls}\NormalTok{())}
\end{Highlighting}
\end{Shaded}

For the Full metropolis hasting we will need to determine the proposal
functions and likelihood functions.

\begin{Shaded}
\begin{Highlighting}[]
\NormalTok{func_q =}\StringTok{ }\ControlFlowTok{function}\NormalTok{(lambda_i, y_i, x_i, beta, sigma)}
\NormalTok{\{}
\NormalTok{  ret =}\StringTok{ }\FloatTok{-0.5}\OperatorTok{*}\KeywordTok{log}\NormalTok{(lambda_i) }\FloatTok{-0.5}\OperatorTok{*}\NormalTok{(lambda_i}\OperatorTok{+}\NormalTok{( (y_i }\OperatorTok{-}\StringTok{ }\NormalTok{x_i }\OperatorTok\StringTok{ }\NormalTok{beta)}\OperatorTok{/}\NormalTok{sigma  )}\OperatorTok{^}\DecValTok{2} \OperatorTok{*}\StringTok{ }\DecValTok{1}\OperatorTok{/}\NormalTok{lambda_i)}
  \KeywordTok{return}\NormalTok{(ret)}
\NormalTok{\}}

\NormalTok{func_F =}\StringTok{ }\ControlFlowTok{function}\NormalTok{(lambda_i, y_i, x_i, beta, sig2)}
\NormalTok{\{}
\NormalTok{  erro =}\StringTok{ }\NormalTok{y_i }\OperatorTok{-}\StringTok{ }\NormalTok{x_i}\OperatorTok\NormalTok{beta}
\NormalTok{  ret =}\StringTok{ }\KeywordTok{dnorm}\NormalTok{(erro, }\DataTypeTok{mean =} \DecValTok{0}\NormalTok{, }\DataTypeTok{sd =}\NormalTok{ (sig2}\OperatorTok{*}\NormalTok{lambda_i)}\OperatorTok{^}\FloatTok{0.5}\NormalTok{, }\DataTypeTok{log =} \OtherTok{TRUE}\NormalTok{) }\OperatorTok{-}\StringTok{ }\NormalTok{lambda_i}\OperatorTok{/}\DecValTok{2} 
  \KeywordTok{return}\NormalTok{(ret)}
\NormalTok{\}}
\end{Highlighting}
\end{Shaded}

We now start with the initial setup. We fix a seed por replication
porposes, and set the initial values of
\(X = \{x_{i1}, x_{i2}, x_{i3}, x_{i4} \}_{i=1}^{n}\)

\begin{Shaded}
\begin{Highlighting}[]
\CommentTok{# Replication Seed}
\KeywordTok{set.seed}\NormalTok{(}\DecValTok{12345}\NormalTok{)}

\CommentTok{# Sample size}
\NormalTok{n =}\StringTok{ }\DecValTok{200}

\CommentTok{# regressors}
\NormalTok{nregress =}\StringTok{ }\DecValTok{4}
\NormalTok{beta_true =}\StringTok{ }\KeywordTok{as.matrix}\NormalTok{(}\DecValTok{0}\OperatorTok{:}\NormalTok{(nregress}\DecValTok{-1}\NormalTok{), nregress,}\DecValTok{1}\NormalTok{)}

\CommentTok{# Regressors Matrix}
\NormalTok{X =}\StringTok{ }\KeywordTok{matrix}\NormalTok{(}\KeywordTok{rnorm}\NormalTok{(nregress}\OperatorTok{*}\NormalTok{n,}\DecValTok{0}\NormalTok{,}\DecValTok{1}\NormalTok{), n, nregress)}

\CommentTok{# Declare of sigma}
\NormalTok{sigma_true =}\StringTok{ }\DecValTok{1}
\end{Highlighting}
\end{Shaded}

We then we proced seting up the simulation. We use a draw form the
double exponential for the erros, however a draw for lambda and then the
construction of the errors could also be used.

\begin{Shaded}
\begin{Highlighting}[]
\CommentTok{# We opt to use the Double exponential extrection}
\NormalTok{USE_DOUBLE_EXP =}\StringTok{ }\OtherTok{TRUE}

\CommentTok{# Simulacao}
\ControlFlowTok{if}\NormalTok{(USE_DOUBLE_EXP)\{}
\NormalTok{  error_true =}\StringTok{ }\NormalTok{nimble}\OperatorTok{::}\KeywordTok{rdexp}\NormalTok{(n, }\DataTypeTok{location =} \DecValTok{0}\NormalTok{, }\DataTypeTok{scale =}\NormalTok{ sigma_true)}
\NormalTok{\} }\ControlFlowTok{else}
\NormalTok{\{}
\NormalTok{  lambda_true =}\StringTok{ }\KeywordTok{rexp}\NormalTok{(n,}\DecValTok{1}\OperatorTok{/}\DecValTok{2}\NormalTok{)}
\NormalTok{  error_true  =}\StringTok{ }\KeywordTok{rnorm}\NormalTok{(n, }\DecValTok{0}\NormalTok{, }\KeywordTok{sqrt}\NormalTok{(sigma_true}\OperatorTok{*}\NormalTok{lambda_true))}
\NormalTok{\}}

\CommentTok{# Simulation}
\NormalTok{Y <-}\StringTok{ }\NormalTok{X}\OperatorTok\NormalTok{beta_true }\OperatorTok{+}\StringTok{ }\NormalTok{error_true}
\end{Highlighting}
\end{Shaded}

Ouw first step is to look at the classical OLS estimator.

\begin{Shaded}
\begin{Highlighting}[]
\CommentTok{# Ols model}
\NormalTok{ols =}\StringTok{ }\KeywordTok{lm}\NormalTok{(Y }\OperatorTok{~}\StringTok{ }\NormalTok{X }\DecValTok{-1}\NormalTok{);}
\NormalTok{sigma2.ols =}\StringTok{ }\KeywordTok{summary}\NormalTok{(ols)}\OperatorTok{$}\NormalTok{sigma}\OperatorTok{^}\DecValTok{2}
\NormalTok{beta.ols =}\StringTok{ }\KeywordTok{matrix}\NormalTok{(ols}\OperatorTok{$}\NormalTok{coefficients, nregress, }\DecValTok{1}\NormalTok{)}
\KeywordTok{summary}\NormalTok{(ols)}
\end{Highlighting}
\end{Shaded}

\begin{verbatim}
## 
## Call:
## lm(formula = Y ~ X - 1)
## 
## Residuals:
##     Min      1Q  Median      3Q     Max 
## -3.6968 -0.7314  0.1168  0.7823  4.7151 
## 
## Coefficients:
##    Estimate Std. Error t value Pr(>|t|)    
## X1 -0.03397    0.08843  -0.384    0.701    
## X2  0.86260    0.09895   8.718  1.2e-15 ***
## X3  1.96746    0.09744  20.191  < 2e-16 ***
## X4  2.93288    0.09408  31.175  < 2e-16 ***
## ---
## Signif. codes:  0 '***' 0.001 '**' 0.01 '*' 0.05 '.' 0.1 ' ' 1
## 
## Residual standard error: 1.338 on 196 degrees of freedom
## Multiple R-squared:  0.8689, Adjusted R-squared:  0.8663 
## F-statistic: 324.9 on 4 and 196 DF,  p-value: < 2.2e-16
\end{verbatim}

\begin{Shaded}
\begin{Highlighting}[]
\CommentTok{# histogram}
\KeywordTok{par}\NormalTok{(}\DataTypeTok{mfrow=}\KeywordTok{c}\NormalTok{(}\DecValTok{1}\NormalTok{,}\DecValTok{2}\NormalTok{))}
\KeywordTok{plot}\NormalTok{(X}\OperatorTok\NormalTok{beta_true, Y, }\DataTypeTok{xlab =} \StringTok{"Y predicted"}\NormalTok{, }\DataTypeTok{ylab =} \StringTok{"Y"}\NormalTok{)}
\KeywordTok{hist}\NormalTok{(ols}\OperatorTok{$}\NormalTok{residuals, }\DataTypeTok{breaks=}\DecValTok{15}\NormalTok{, }\DataTypeTok{main=}\StringTok{"Histogram of Ols residuals"}\NormalTok{, }\DataTypeTok{xlab =} \StringTok{"Residuals"}\NormalTok{, }\DataTypeTok{ylab =} \StringTok{"Frequency"}\NormalTok{)}
\end{Highlighting}
\end{Shaded}

\includegraphics{TakeHome_MarkDown_files/figure-latex/OLS-1.pdf}

Now we make the initial setup of the MCMC. We set a \emph{Burn up}
sample of size \(10^4\), and a final sample of size \(10^3\). We also
store the draws in a table.

\begin{Shaded}
\begin{Highlighting}[]
\CommentTok{# MCMC set-up}
\NormalTok{M0    =}\StringTok{ }\DecValTok{1000}  \CommentTok{# Final}
\NormalTok{M     =}\StringTok{ }\DecValTok{10000} \CommentTok{# Burn up}
\NormalTok{niter =}\StringTok{ }\NormalTok{M0}\OperatorTok{+}\NormalTok{M}


\CommentTok{# TABLE DRAWS}
\NormalTok{ncol.draws =}\StringTok{ }\DecValTok{1} \OperatorTok{+}\StringTok{ }\NormalTok{nregress}
\NormalTok{draws.mc =}\StringTok{ }\KeywordTok{matrix}\NormalTok{(}\DecValTok{0}\NormalTok{, }\DataTypeTok{nrow =}\NormalTok{ niter, }\DataTypeTok{ncol =}\NormalTok{ ncol.draws)}
\KeywordTok{colnames}\NormalTok{(draws.mc) =}\StringTok{ }\KeywordTok{c}\NormalTok{(}\StringTok{"sigma"}\NormalTok{, }\KeywordTok{paste}\NormalTok{(}\StringTok{"beta"}\NormalTok{, }\DecValTok{1}\OperatorTok{:}\NormalTok{nregress))}

\NormalTok{draws.gibbs =}\StringTok{ }\KeywordTok{matrix}\NormalTok{(}\DecValTok{0}\NormalTok{, }\DataTypeTok{nrow =}\NormalTok{ niter, }\DataTypeTok{ncol =}\NormalTok{ ncol.draws)}
\KeywordTok{colnames}\NormalTok{(draws.gibbs) =}\StringTok{ }\KeywordTok{c}\NormalTok{(}\StringTok{"sigma"}\NormalTok{, }\KeywordTok{paste}\NormalTok{(}\StringTok{"beta"}\NormalTok{, }\DecValTok{1}\OperatorTok{:}\NormalTok{nregress))}
\end{Highlighting}
\end{Shaded}

For the Priors we will set the following. we will assume that
\(\beta_0 = 0\), this impplies that we do not expect \(Y\) to be
correlated with \(X\), however we are insecure about this fact and
assume a standard deviation of 5 for each beta
(\(V_0 = diag(25)_{n=4}\)). For \(\sigma^2\) we will set the priors
\(\sigma_0 = 1\) and \(\nu_0 = 2.5\). We also set the initial values for
our interations.

\begin{Shaded}
\begin{Highlighting}[]
\CommentTok{# priors of beta}
\NormalTok{beta_}\DecValTok{0}\NormalTok{ =}\StringTok{ }\KeywordTok{matrix}\NormalTok{(}\DecValTok{0}\NormalTok{, nregress, }\DecValTok{1}\NormalTok{)}
\NormalTok{V_}\DecValTok{0}\NormalTok{ =}\StringTok{ }\KeywordTok{diag}\NormalTok{(}\DecValTok{25}\NormalTok{, nregress)}

\CommentTok{# priors of sigma}
\NormalTok{sigma2_}\DecValTok{0}\NormalTok{ =}\StringTok{ }\DecValTok{1}
\NormalTok{nu_}\DecValTok{0}\NormalTok{ =}\StringTok{ }\FloatTok{2.5}

\CommentTok{# initial Values}
\NormalTok{sigma2 =}\StringTok{ }\NormalTok{sigma2.ols}
\NormalTok{beta =}\StringTok{ }\NormalTok{beta.ols}
\NormalTok{lambda =}\StringTok{ }\KeywordTok{rep}\NormalTok{(}\DecValTok{1}\NormalTok{,n)}
\end{Highlighting}
\end{Shaded}

Then we proceed with the MCMC, one could make the draw from the inverse
Gaussian using the \emph{statmod} r package or the procedure done by
michael et. al.

\begin{Shaded}
\begin{Highlighting}[]
\CommentTok{# We opt to use Michael method}
\NormalTok{USE_MICHAEL_METHOD =}\StringTok{ }\OtherTok{TRUE}

\CommentTok{# table to store execution time}
\NormalTok{dt_time =}\StringTok{ }\KeywordTok{data.frame}\NormalTok{(}\DataTypeTok{Method =} \KeywordTok{c}\NormalTok{(}\StringTok{"MH"}\NormalTok{, }\StringTok{"GIBBS"}\NormalTok{), }\DataTypeTok{Time =} \OtherTok{NA}\NormalTok{, }\DataTypeTok{ess=}\OtherTok{NA}\NormalTok{, }\DataTypeTok{ess_ps =} \OtherTok{NA}\NormalTok{)}
\ControlFlowTok{for}\NormalTok{ (k }\ControlFlowTok{in} \DecValTok{1}\OperatorTok{:}\DecValTok{2}\NormalTok{) \{}
  \ControlFlowTok{if}\NormalTok{(k}\OperatorTok{==}\DecValTok{1}\NormalTok{)}
\NormalTok{  \{ MH =}\StringTok{ }\OtherTok{TRUE}\NormalTok{ \}}
  \ControlFlowTok{else}
\NormalTok{  \{ MH=}\OtherTok{FALSE}\NormalTok{ \}}
  
\NormalTok{  tictoc}\OperatorTok{::}\KeywordTok{tic}\NormalTok{(MH)}
  \ControlFlowTok{for}\NormalTok{ (i }\ControlFlowTok{in} \DecValTok{1}\OperatorTok{:}\NormalTok{(niter))\{}
    
    \CommentTok{# Inicialize Lambda^\{-1\}  matrix}
\NormalTok{    Lambda_}\DecValTok{1}\NormalTok{ =}\StringTok{ }\KeywordTok{solve}\NormalTok{(}\KeywordTok{diag}\NormalTok{(lambda))}
    
    \CommentTok{# full conditional of sigma2  }
\NormalTok{    d0=(nu_}\DecValTok{0} \OperatorTok{*}\StringTok{ }\NormalTok{sigma2_}\DecValTok{0}\NormalTok{)}\OperatorTok{/}\DecValTok{2}
\NormalTok{    par1 =}\StringTok{ }\NormalTok{(nu_}\DecValTok{0} \OperatorTok{+}\StringTok{ }\NormalTok{n)}\OperatorTok{/}\DecValTok{2}
\NormalTok{    par2 =}\StringTok{ }\NormalTok{d0 }\OperatorTok{+}\StringTok{ }\NormalTok{( }\KeywordTok{t}\NormalTok{(Y}\OperatorTok{-}\NormalTok{X}\OperatorTok\NormalTok{beta) }\OperatorTok\StringTok{ }\NormalTok{Lambda_}\DecValTok{1} \OperatorTok\StringTok{ }\NormalTok{(Y}\OperatorTok{-}\NormalTok{X}\OperatorTok\NormalTok{beta) )}\OperatorTok{/}\DecValTok{2}
    
    \CommentTok{# Conditional distribution of sigma }
\NormalTok{    sig2 =}\StringTok{ }\DecValTok{1}\OperatorTok{/}\KeywordTok{rgamma}\NormalTok{(}\DecValTok{1}\NormalTok{, par1, par2)}
    
    \CommentTok{# full conditional of beta}
\NormalTok{    XtX =}\StringTok{ }\KeywordTok{t}\NormalTok{(X) }\OperatorTok\StringTok{ }\NormalTok{Lambda_}\DecValTok{1} \OperatorTok\StringTok{ }\NormalTok{X}
\NormalTok{    XtY =}\StringTok{ }\KeywordTok{t}\NormalTok{(X) }\OperatorTok\StringTok{ }\NormalTok{Lambda_}\DecValTok{1} \OperatorTok\StringTok{ }\NormalTok{Y}
    
\NormalTok{    V_}\DecValTok{1}\NormalTok{   =}\StringTok{ }\KeywordTok{solve}\NormalTok{(XtX}\OperatorTok{/}\NormalTok{sig2 }\OperatorTok{+}\StringTok{ }\KeywordTok{solve}\NormalTok{(V_}\DecValTok{0}\NormalTok{))}
\NormalTok{    beta_}\DecValTok{1}\NormalTok{ =}\StringTok{ }\NormalTok{V_}\DecValTok{1} \OperatorTok\StringTok{ }\NormalTok{(XtY}\OperatorTok{/}\NormalTok{sig2 }\OperatorTok{+}\StringTok{ }\KeywordTok{solve}\NormalTok{(V_}\DecValTok{0}\NormalTok{) }\OperatorTok\StringTok{ }\NormalTok{beta_}\DecValTok{0}\NormalTok{)}
\NormalTok{    beta =}\StringTok{  }\NormalTok{beta_}\DecValTok{1} \OperatorTok{+}\StringTok{ }\KeywordTok{t}\NormalTok{(}\KeywordTok{chol}\NormalTok{(V_}\DecValTok{1}\NormalTok{)) }\OperatorTok\StringTok{ }\KeywordTok{rnorm}\NormalTok{(nregress)}
    
    \CommentTok{# Foolowing Koop. (Bayesian Econometrics Methods) pag 260}
    
    \CommentTok{# Draw of lambda}
    \ControlFlowTok{for}\NormalTok{ (j }\ControlFlowTok{in} \DecValTok{1}\OperatorTok{:}\NormalTok{n)\{}
\NormalTok{      y_j =}\StringTok{ }\NormalTok{Y[j, }\DecValTok{1}\NormalTok{]}
\NormalTok{      x_j =}\StringTok{ }\NormalTok{X[j, ]}
      
      \CommentTok{# draw de nu_0}
\NormalTok{      nu_michael_}\DecValTok{0}\NormalTok{ =}\StringTok{ }\KeywordTok{rchisq}\NormalTok{(}\DecValTok{1}\NormalTok{,}\DecValTok{1}\NormalTok{)}
      
      \CommentTok{# mu for row j}
\NormalTok{      mu_j =}\StringTok{ }\KeywordTok{abs}\NormalTok{(}\KeywordTok{sqrt}\NormalTok{(sig2)}\OperatorTok{/}\NormalTok{(y_j }\OperatorTok{-}\StringTok{ }\NormalTok{x_j }\OperatorTok\StringTok{ }\NormalTok{beta)) }
      
      \ControlFlowTok{if}\NormalTok{(USE_MICHAEL_METHOD)\{}
        \CommentTok{#  x1 e x2}
\NormalTok{        x_}\DecValTok{1}\NormalTok{ =}\StringTok{ }\NormalTok{mu_j }\OperatorTok{+}\StringTok{ }\NormalTok{(mu_j}\OperatorTok{^}\DecValTok{2} \OperatorTok{*}\StringTok{ }\NormalTok{nu_michael_}\DecValTok{0}\NormalTok{)}\OperatorTok{/}\DecValTok{2} \OperatorTok{-}\StringTok{ }\NormalTok{mu_j}\OperatorTok{/}\DecValTok{2} \OperatorTok{*}\StringTok{ }\NormalTok{(}\DecValTok{4} \OperatorTok{*}\StringTok{ }\NormalTok{mu_j }\OperatorTok{*}\StringTok{ }\NormalTok{nu_michael_}\DecValTok{0} \OperatorTok{+}\StringTok{ }\NormalTok{mu_j}\OperatorTok{^}\DecValTok{2} \OperatorTok{*}\StringTok{ }\NormalTok{nu_michael_}\DecValTok{0}\OperatorTok{^}\DecValTok{2}\NormalTok{)}\OperatorTok{^}\FloatTok{0.5}
\NormalTok{        x_}\DecValTok{2}\NormalTok{ =}\StringTok{ }\NormalTok{mu_j}\OperatorTok{^}\DecValTok{2} \OperatorTok{/}\StringTok{ }\NormalTok{x_}\DecValTok{1}
        
        \CommentTok{# decide between x_1 and x_2}
\NormalTok{        p.treshold =}\StringTok{ }\NormalTok{mu_j}\OperatorTok{/}\NormalTok{(mu_j }\OperatorTok{+}\StringTok{ }\NormalTok{x_}\DecValTok{1}\NormalTok{)}
        \ControlFlowTok{if}\NormalTok{ (}\KeywordTok{runif}\NormalTok{(}\DecValTok{1}\NormalTok{) }\OperatorTok{<}\StringTok{ }\NormalTok{p.treshold)}
\NormalTok{        \{}
\NormalTok{          x_star =}\StringTok{ }\NormalTok{x_}\DecValTok{1}
\NormalTok{        \}}
        \ControlFlowTok{else}
\NormalTok{        \{}
\NormalTok{          x_star =}\StringTok{ }\NormalTok{x_}\DecValTok{2}
\NormalTok{        \}}
        
        \CommentTok{# invert x_star}
\NormalTok{        lambda_j =}\StringTok{ }\DecValTok{1}\OperatorTok{/}\NormalTok{x_star}
\NormalTok{      \}}
      \ControlFlowTok{else}
\NormalTok{      \{}
\NormalTok{        lambda_j =}\StringTok{ }\NormalTok{statmod}\OperatorTok{::}\KeywordTok{rinvgauss}\NormalTok{(}\DecValTok{1}\NormalTok{, mu_j, }\DataTypeTok{shape =} \DecValTok{1}\NormalTok{) }
\NormalTok{      \}}
      
      
      \CommentTok{# now the Metropolis-Hasting}
      \ControlFlowTok{if}\NormalTok{ ((lambda_j }\OperatorTok{>}\DecValTok{0}\NormalTok{) }\OperatorTok{&}\StringTok{ }\NormalTok{(MH))\{}
        
\NormalTok{        deno =}\StringTok{ }\KeywordTok{func_F}\NormalTok{(lambda_j, y_j, x_j, beta, sig2) }\OperatorTok{+}\StringTok{ }\KeywordTok{func_q}\NormalTok{(lambda[j], y_j, x_j, beta, }\KeywordTok{sqrt}\NormalTok{(sig2))}
\NormalTok{        nume =}\StringTok{ }\KeywordTok{func_F}\NormalTok{(lambda[j], y_j, x_j, beta, sig2) }\OperatorTok{+}\StringTok{ }\KeywordTok{func_q}\NormalTok{(lambda_j, y_j, x_j, beta, }\KeywordTok{sqrt}\NormalTok{(sig2))}
        
\NormalTok{        log.rho =}\StringTok{ }\KeywordTok{min}\NormalTok{(}\DecValTok{0}\NormalTok{, nume}\OperatorTok{-}\NormalTok{deno)}
        \ControlFlowTok{if}\NormalTok{ (}\KeywordTok{log}\NormalTok{(}\KeywordTok{runif}\NormalTok{(}\DecValTok{1}\NormalTok{)) }\OperatorTok{<}\StringTok{ }\NormalTok{log.rho)\{}
\NormalTok{          lambda[j] =}\StringTok{ }\NormalTok{lambda_j}
\NormalTok{        \}}
\NormalTok{      \} }
      \ControlFlowTok{else} \ControlFlowTok{if}\NormalTok{ (}\OperatorTok{!}\NormalTok{MH)}
\NormalTok{      \{}
\NormalTok{        lambda[j] =}\StringTok{ }\NormalTok{lambda_j}
\NormalTok{      \}}
\NormalTok{    \}}
    
    \CommentTok{# storing draws}
    \ControlFlowTok{if}\NormalTok{(MH)}
\NormalTok{    \{draws.mc[i,] =}\StringTok{ }\KeywordTok{c}\NormalTok{(sig2, beta)\}}
    \ControlFlowTok{else}
\NormalTok{    \{draws.gibbs[i,] =}\StringTok{ }\KeywordTok{c}\NormalTok{(sig2, beta)\}}
    
\NormalTok{  \}}
  
  \CommentTok{# Determine Execution Time}
\NormalTok{  exec_time =}\StringTok{ }\NormalTok{tictoc}\OperatorTok{::}\KeywordTok{toc}\NormalTok{()}
\NormalTok{  dt_time[k, }\StringTok{"Time"}\NormalTok{] =}\StringTok{ }\NormalTok{exec_time}\OperatorTok{$}\NormalTok{toc }\OperatorTok{-}\StringTok{ }\NormalTok{exec_time}\OperatorTok{$}\NormalTok{tic}
  \KeywordTok{rm}\NormalTok{(}\DataTypeTok{list =} \KeywordTok{c}\NormalTok{(}\StringTok{"exec_time"}\NormalTok{))}
  
  \CommentTok{# Determine ESS}
  \ControlFlowTok{if}\NormalTok{(MH)}
\NormalTok{  \{}
\NormalTok{    dt_time[k, }\StringTok{"ess"}\NormalTok{] =}\StringTok{ }\KeywordTok{round}\NormalTok{(M}\OperatorTok{/}\NormalTok{(}\DecValTok{1}\OperatorTok{+}\DecValTok{2}\OperatorTok{*}\KeywordTok{sum}\NormalTok{(}\KeywordTok{acf}\NormalTok{(draws.mc[,}\StringTok{"sigma"}\NormalTok{],}\DataTypeTok{lag.max=}\DecValTok{1000}\NormalTok{,}\DataTypeTok{plot=}\OtherTok{FALSE}\NormalTok{)}\OperatorTok{$}\NormalTok{acf[}\DecValTok{2}\OperatorTok{:}\DecValTok{1001}\NormalTok{])))}
\NormalTok{  \}}
  \ControlFlowTok{else}
\NormalTok{  \{}
\NormalTok{    dt_time[k, }\StringTok{"ess"}\NormalTok{] =}\StringTok{ }\KeywordTok{round}\NormalTok{(M}\OperatorTok{/}\NormalTok{(}\DecValTok{1}\OperatorTok{+}\DecValTok{2}\OperatorTok{*}\KeywordTok{sum}\NormalTok{(}\KeywordTok{acf}\NormalTok{(draws.gibbs[,}\StringTok{"sigma"}\NormalTok{],}\DataTypeTok{lag.max=}\DecValTok{1000}\NormalTok{,}\DataTypeTok{plot=}\OtherTok{FALSE}\NormalTok{)}\OperatorTok{$}\NormalTok{acf[}\DecValTok{2}\OperatorTok{:}\DecValTok{1001}\NormalTok{])))}
\NormalTok{  \}}
  
\NormalTok{\}}
\end{Highlighting}
\end{Shaded}

\begin{verbatim}
## TRUE: 85.85 sec elapsed
## FALSE: 47.59 sec elapsed
\end{verbatim}

\begin{Shaded}
\begin{Highlighting}[]
\CommentTok{# Determine the ess per second}
\NormalTok{dt_time}\OperatorTok{$}\NormalTok{ess_ps =}\StringTok{ }\NormalTok{dt_time}\OperatorTok{$}\NormalTok{ess }\OperatorTok{/}\StringTok{ }\NormalTok{dt_time}\OperatorTok{$}\NormalTok{Time}
\end{Highlighting}
\end{Shaded}

We them print the distributions

\begin{Shaded}
\begin{Highlighting}[]
\NormalTok{draws2 =}\StringTok{ }\KeywordTok{data.frame}\NormalTok{(draws.mc[(M0}\OperatorTok{+}\DecValTok{1}\NormalTok{)}\OperatorTok{:}\NormalTok{niter,])}
\KeywordTok{colnames}\NormalTok{(draws2)}
\end{Highlighting}
\end{Shaded}

\begin{verbatim}
## [1] "sigma"  "beta.1" "beta.2" "beta.3" "beta.4"
\end{verbatim}

\begin{Shaded}
\begin{Highlighting}[]
\KeywordTok{summary}\NormalTok{(draws2}\OperatorTok{$}\NormalTok{sigma)}
\end{Highlighting}
\end{Shaded}

\begin{verbatim}
##    Min. 1st Qu.  Median    Mean 3rd Qu.    Max. 
##  0.5986  0.9145  1.0045  1.0172  1.1057  1.8359
\end{verbatim}

\begin{Shaded}
\begin{Highlighting}[]
\KeywordTok{par}\NormalTok{(}\DataTypeTok{mfrow=}\KeywordTok{c}\NormalTok{(}\DecValTok{1}\NormalTok{,}\DecValTok{2}\NormalTok{))}
\KeywordTok{hist}\NormalTok{(draws2}\OperatorTok{$}\NormalTok{sigma, }\DataTypeTok{breaks =} \DecValTok{50}\NormalTok{, }\DataTypeTok{main=}\StringTok{"Histogram of Sigma"}\NormalTok{, }\DataTypeTok{xlab =} \StringTok{"Sigma"}\NormalTok{, }\DataTypeTok{ylab =} \StringTok{"Frequency"}\NormalTok{, }\DataTypeTok{xlim =} \KeywordTok{range}\NormalTok{(sigma_true, sigma2.ols, draws2}\OperatorTok{$}\NormalTok{sigma))}
\KeywordTok{abline}\NormalTok{(}\DataTypeTok{v=}\NormalTok{sigma_true, }\DataTypeTok{col=}\StringTok{"red"}\NormalTok{)}
\KeywordTok{abline}\NormalTok{(}\DataTypeTok{v=}\NormalTok{sigma2.ols, }\DataTypeTok{col=}\StringTok{"blue"}\NormalTok{)}
\KeywordTok{legend}\NormalTok{(}\StringTok{"bottom"}\NormalTok{, }\KeywordTok{c}\NormalTok{(}\StringTok{"OLS"}\NormalTok{, }\StringTok{"TRUE"}\NormalTok{), }\DataTypeTok{col =} \KeywordTok{c}\NormalTok{(}\StringTok{"blue"}\NormalTok{, }\StringTok{"red"}\NormalTok{), }\DataTypeTok{lty=}\DecValTok{1}\NormalTok{)}
\KeywordTok{acf}\NormalTok{(draws2}\OperatorTok{$}\NormalTok{sigma)}
\end{Highlighting}
\end{Shaded}

\includegraphics{TakeHome_MarkDown_files/figure-latex/unnamed-chunk-2-1.pdf}

\begin{Shaded}
\begin{Highlighting}[]
\KeywordTok{par}\NormalTok{(}\DataTypeTok{mfrow=}\KeywordTok{c}\NormalTok{(}\DecValTok{1}\NormalTok{,}\DecValTok{2}\NormalTok{))}
\KeywordTok{hist}\NormalTok{(draws2}\OperatorTok{$}\NormalTok{beta}\FloatTok{.1}\NormalTok{, }\DataTypeTok{breaks =} \DecValTok{50}\NormalTok{, }\DataTypeTok{main=}\StringTok{"Histogram of beta_1"}\NormalTok{, }\DataTypeTok{xlab =} \StringTok{"beta_1"}\NormalTok{, }\DataTypeTok{ylab =} \StringTok{"Frequency"}\NormalTok{, }\DataTypeTok{xlim =} \KeywordTok{range}\NormalTok{(beta_true[}\DecValTok{1}\NormalTok{], beta.ols[}\DecValTok{1}\NormalTok{], draws2}\OperatorTok{$}\NormalTok{beta}\FloatTok{.1}\NormalTok{))}
\KeywordTok{abline}\NormalTok{(}\DataTypeTok{v =}\NormalTok{ beta_true[}\DecValTok{1}\NormalTok{], }\DataTypeTok{col=}\StringTok{"red"}\NormalTok{)}
\KeywordTok{abline}\NormalTok{(}\DataTypeTok{v =}\NormalTok{ beta.ols[}\DecValTok{1}\NormalTok{], }\DataTypeTok{col=}\StringTok{"blue"}\NormalTok{)}
\KeywordTok{legend}\NormalTok{(}\StringTok{"bottom"}\NormalTok{, }\KeywordTok{c}\NormalTok{(}\StringTok{"OLS"}\NormalTok{, }\StringTok{"TRUE"}\NormalTok{), }\DataTypeTok{col =} \KeywordTok{c}\NormalTok{(}\StringTok{"blue"}\NormalTok{, }\StringTok{"red"}\NormalTok{), }\DataTypeTok{lty=}\DecValTok{1}\NormalTok{)}
\KeywordTok{plot}\NormalTok{(draws2}\OperatorTok{$}\NormalTok{beta}\FloatTok{.1}\NormalTok{, }\DataTypeTok{xlab=}\StringTok{"Iteration"}\NormalTok{, }\DataTypeTok{ylab=}\StringTok{"beta_1"}\NormalTok{,}\DataTypeTok{main=}\StringTok{"Interations"}\NormalTok{, }\DataTypeTok{type =} \StringTok{"l"}\NormalTok{)}
\end{Highlighting}
\end{Shaded}

\includegraphics{TakeHome_MarkDown_files/figure-latex/unnamed-chunk-2-2.pdf}

\begin{Shaded}
\begin{Highlighting}[]
\KeywordTok{par}\NormalTok{(}\DataTypeTok{mfrow=}\KeywordTok{c}\NormalTok{(}\DecValTok{1}\NormalTok{,}\DecValTok{2}\NormalTok{))}
\KeywordTok{hist}\NormalTok{(draws2}\OperatorTok{$}\NormalTok{beta}\FloatTok{.2}\NormalTok{, }\DataTypeTok{breaks =} \DecValTok{50}\NormalTok{, }\DataTypeTok{main=}\StringTok{"Histogram of beta_2"}\NormalTok{, }\DataTypeTok{xlab =} \StringTok{"beta_2"}\NormalTok{, }\DataTypeTok{ylab =} \StringTok{"Frequency"}\NormalTok{, }\DataTypeTok{xlim =} \KeywordTok{range}\NormalTok{(beta_true[}\DecValTok{2}\NormalTok{], beta.ols[}\DecValTok{2}\NormalTok{], draws2}\OperatorTok{$}\NormalTok{beta}\FloatTok{.2}\NormalTok{))}
\KeywordTok{abline}\NormalTok{(}\DataTypeTok{v =}\NormalTok{ beta_true[}\DecValTok{2}\NormalTok{], }\DataTypeTok{col=}\StringTok{"red"}\NormalTok{)}
\KeywordTok{abline}\NormalTok{(}\DataTypeTok{v =}\NormalTok{ ols}\OperatorTok{$}\NormalTok{coefficients[}\DecValTok{2}\NormalTok{], }\DataTypeTok{col=}\StringTok{"blue"}\NormalTok{)}
\KeywordTok{legend}\NormalTok{(}\StringTok{"bottom"}\NormalTok{, }\KeywordTok{c}\NormalTok{(}\StringTok{"OLS"}\NormalTok{, }\StringTok{"TRUE"}\NormalTok{), }\DataTypeTok{col =} \KeywordTok{c}\NormalTok{(}\StringTok{"blue"}\NormalTok{, }\StringTok{"red"}\NormalTok{), }\DataTypeTok{lty=}\DecValTok{1}\NormalTok{)}
\KeywordTok{plot}\NormalTok{(draws2}\OperatorTok{$}\NormalTok{beta}\FloatTok{.2}\NormalTok{, }\DataTypeTok{xlab=}\StringTok{"Iteration"}\NormalTok{, }\DataTypeTok{ylab=}\StringTok{"beta_2"}\NormalTok{,}\DataTypeTok{main=}\StringTok{"Interations"}\NormalTok{, }\DataTypeTok{type =} \StringTok{"l"}\NormalTok{)}
\end{Highlighting}
\end{Shaded}

\includegraphics{TakeHome_MarkDown_files/figure-latex/unnamed-chunk-2-3.pdf}

\begin{Shaded}
\begin{Highlighting}[]
\KeywordTok{par}\NormalTok{(}\DataTypeTok{mfrow=}\KeywordTok{c}\NormalTok{(}\DecValTok{1}\NormalTok{,}\DecValTok{2}\NormalTok{))}
\KeywordTok{hist}\NormalTok{(draws2}\OperatorTok{$}\NormalTok{beta}\FloatTok{.3}\NormalTok{, }\DataTypeTok{breaks =} \DecValTok{50}\NormalTok{, }\DataTypeTok{main=}\StringTok{"Histogram of beta_3"}\NormalTok{, }\DataTypeTok{xlab =} \StringTok{"beta_3"}\NormalTok{, }\DataTypeTok{ylab =} \StringTok{"Frequency"}\NormalTok{, }\DataTypeTok{xlim =} \KeywordTok{range}\NormalTok{(beta_true[}\DecValTok{3}\NormalTok{], beta.ols[}\DecValTok{3}\NormalTok{], draws2}\OperatorTok{$}\NormalTok{beta}\FloatTok{.3}\NormalTok{))}
\KeywordTok{abline}\NormalTok{(}\DataTypeTok{v =}\NormalTok{ beta_true[}\DecValTok{3}\NormalTok{], }\DataTypeTok{col=}\StringTok{"red"}\NormalTok{)}
\KeywordTok{abline}\NormalTok{(}\DataTypeTok{v =}\NormalTok{ ols}\OperatorTok{$}\NormalTok{coefficients[}\DecValTok{3}\NormalTok{], }\DataTypeTok{col=}\StringTok{"blue"}\NormalTok{)}
\KeywordTok{legend}\NormalTok{(}\StringTok{"bottom"}\NormalTok{, }\KeywordTok{c}\NormalTok{(}\StringTok{"OLS"}\NormalTok{, }\StringTok{"TRUE"}\NormalTok{), }\DataTypeTok{col =} \KeywordTok{c}\NormalTok{(}\StringTok{"blue"}\NormalTok{, }\StringTok{"red"}\NormalTok{), }\DataTypeTok{lty=}\DecValTok{1}\NormalTok{)}
\KeywordTok{plot}\NormalTok{(draws2}\OperatorTok{$}\NormalTok{beta}\FloatTok{.3}\NormalTok{, }\DataTypeTok{xlab=}\StringTok{"Iteration"}\NormalTok{, }\DataTypeTok{ylab=}\StringTok{"beta_3"}\NormalTok{,}\DataTypeTok{main=}\StringTok{"Interations"}\NormalTok{, }\DataTypeTok{type =} \StringTok{"l"}\NormalTok{)}
\end{Highlighting}
\end{Shaded}

\includegraphics{TakeHome_MarkDown_files/figure-latex/unnamed-chunk-2-4.pdf}

\begin{Shaded}
\begin{Highlighting}[]
\KeywordTok{par}\NormalTok{(}\DataTypeTok{mfrow=}\KeywordTok{c}\NormalTok{(}\DecValTok{1}\NormalTok{,}\DecValTok{2}\NormalTok{))}
\KeywordTok{hist}\NormalTok{(draws2}\OperatorTok{$}\NormalTok{beta}\FloatTok{.4}\NormalTok{, }\DataTypeTok{breaks =} \DecValTok{50}\NormalTok{, }\DataTypeTok{main=}\StringTok{"Histogram of beta_4"}\NormalTok{, }\DataTypeTok{xlab =} \StringTok{"beta_4"}\NormalTok{, }\DataTypeTok{ylab =} \StringTok{"Frequency"}\NormalTok{, }\DataTypeTok{xlim =} \KeywordTok{range}\NormalTok{(beta_true[}\DecValTok{4}\NormalTok{], beta.ols[}\DecValTok{4}\NormalTok{], draws2}\OperatorTok{$}\NormalTok{beta}\FloatTok{.4}\NormalTok{))}
\KeywordTok{abline}\NormalTok{(}\DataTypeTok{v =}\NormalTok{ beta_true[}\DecValTok{4}\NormalTok{], }\DataTypeTok{col=}\StringTok{"red"}\NormalTok{)}
\KeywordTok{abline}\NormalTok{(}\DataTypeTok{v =}\NormalTok{ ols}\OperatorTok{$}\NormalTok{coefficients[}\DecValTok{4}\NormalTok{], }\DataTypeTok{col=}\StringTok{"blue"}\NormalTok{)}
\KeywordTok{legend}\NormalTok{(}\StringTok{"bottom"}\NormalTok{, }\KeywordTok{c}\NormalTok{(}\StringTok{"OLS"}\NormalTok{, }\StringTok{"TRUE"}\NormalTok{), }\DataTypeTok{col =} \KeywordTok{c}\NormalTok{(}\StringTok{"blue"}\NormalTok{, }\StringTok{"red"}\NormalTok{), }\DataTypeTok{lty=}\DecValTok{1}\NormalTok{)}
\KeywordTok{plot}\NormalTok{(draws2}\OperatorTok{$}\NormalTok{beta}\FloatTok{.3}\NormalTok{, }\DataTypeTok{xlab=}\StringTok{"Iteration"}\NormalTok{, }\DataTypeTok{ylab=}\StringTok{"beta_4"}\NormalTok{,}\DataTypeTok{main=}\StringTok{"Interations"}\NormalTok{, }\DataTypeTok{type =} \StringTok{"l"}\NormalTok{)}
\end{Highlighting}
\end{Shaded}

\includegraphics{TakeHome_MarkDown_files/figure-latex/unnamed-chunk-2-5.pdf}

Lastily we evaluate if MH perform better. We present the following
table.

\begin{Shaded}
\begin{Highlighting}[]
\NormalTok{dt_time}
\end{Highlighting}
\end{Shaded}

\begin{verbatim}
##   Method  Time   ess   ess_ps
## 1     MH 85.85 46475 541.3512
## 2  GIBBS 47.59  7443 156.3984
\end{verbatim}

\end{document}
